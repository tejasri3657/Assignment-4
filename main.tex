\documentclass[journal,12pt,twocolumn]{IEEEtran}

\usepackage{setspace}
\usepackage{gensymb}

\singlespacing


\usepackage[cmex10]{amsmath}

\usepackage{amsthm}

\usepackage{mathrsfs}
\usepackage{txfonts}
\usepackage{stfloats}
\usepackage{bm}
\usepackage{cite}
\usepackage{cases}
\usepackage{subfig}

\usepackage{longtable}
\usepackage{multirow}

\usepackage{enumitem}
\usepackage{mathtools}
\usepackage{steinmetz}
\usepackage{tikz}
\usepackage{circuitikz}
\usepackage{verbatim}
\usepackage{tfrupee}
\usepackage[breaklinks=true]{hyperref}
\usepackage{graphicx}
\usepackage{tkz-euclide}
\usepackage{float}

\usetikzlibrary{calc,math}
\usepackage{listings}
    \usepackage{color}                                            %%
    \usepackage{array}                                            %%
    \usepackage{longtable}                                        %%
    \usepackage{calc}                                             %%
    \usepackage{multirow}                                         %%
    \usepackage{hhline}                                           %%
    \usepackage{ifthen}                                           %%
    \usepackage{lscape}     
\usepackage{multicol}
\usepackage{chngcntr}

\DeclareMathOperator*{\Res}{Res}

\renewcommand\thesection{\arabic{section}}
\renewcommand\thesubsection{\thesection.\arabic{subsection}}
\renewcommand\thesubsubsection{\thesubsection.\arabic{subsubsection}}

\renewcommand\thesectiondis{\arabic{section}}
\renewcommand\thesubsectiondis{\thesectiondis.\arabic{subsection}}
\renewcommand\thesubsubsectiondis{\thesubsectiondis.\arabic{subsubsection}}


\hyphenation{op-tical net-works semi-conduc-tor}
\def\inputGnumericTable{}                                 %%

\lstset{
%language=C,
frame=single, 
breaklines=true,
columns=fullflexible
}
\begin{document}
\newtheorem{theorem}{Theorem}[section]
\newtheorem{problem}{Problem}
\newtheorem{proposition}{Proposition}[section]
\newtheorem{lemma}{Lemma}[section]
\newtheorem{corollary}[theorem]{Corollary}
\newtheorem{example}{Example}[section]
\newtheorem{definition}[problem]{Definition}

\newcommand{\BEQA}{\begin{eqnarray}}
\newcommand{\EEQA}{\end{eqnarray}}
\newcommand{\define}{\stackrel{\triangle}{=}}
\bibliographystyle{IEEEtran}
\providecommand{\mbf}{\mathbf}
\providecommand{\pr}[1]{\ensuremath{\Pr\left(#1\right)}}
\providecommand{\qfunc}[1]{\ensuremath{Q\left(#1\right)}}
\providecommand{\sbrak}[1]{\ensuremath{{}\left[#1\right]}}
\providecommand{\lsbrak}[1]{\ensuremath{{}\left[#1\right.}}
\providecommand{\rsbrak}[1]{\ensuremath{{}\left.#1\right]}}
\providecommand{\brak}[1]{\ensuremath{\left(#1\right)}}
\providecommand{\lbrak}[1]{\ensuremath{\left(#1\right.}}
\providecommand{\rbrak}[1]{\ensuremath{\left.#1\right)}}
\providecommand{\cbrak}[1]{\ensuremath{\left\{#1\right\}}}
\providecommand{\lcbrak}[1]{\ensuremath{\left\{#1\right.}}
\providecommand{\rcbrak}[1]{\ensuremath{\left.#1\right\}}}
\theoremstyle{remark}
\newtheorem{rem}{Remark}
\newcommand{\sgn}{\mathop{\mathrm{sgn}}}
\providecommand{\abs}[1]{\left\vert#1\right\vert}
\providecommand{\res}[1]{\Res\displaylimits_{#1}} 
\providecommand{\norm}[1]{\left\lVert#1\right\rVert}
%\providecommand{\norm}[1]{\lVert#1\rVert}
\providecommand{\mtx}[1]{\mathbf{#1}}
\providecommand{\mean}[1]{E\left[ #1 \right]}
\providecommand{\fourier}{\overset{\mathcal{F}}{ \rightleftharpoons}}
%\providecommand{\hilbert}{\overset{\mathcal{H}}{ \rightleftharpoons}}
\providecommand{\system}{\overset{\mathcal{H}}{ \longleftrightarrow}}
	%\newcommand{\solution}[2]{\textbf{Solution:}{#1}}
\newcommand{\solution}{\noindent \textbf{Solution: }}
\newcommand{\cosec}{\,\text{cosec}\,}
\providecommand{\dec}[2]{\ensuremath{\overset{#1}{\underset{#2}{\gtrless}}}}
\newcommand{\myvec}[1]{\ensuremath{\begin{pmatrix}#1\end{pmatrix}}}
\newcommand{\mydet}[1]{\ensuremath{\begin{vmatrix}#1\end{vmatrix}}}
\numberwithin{equation}{subsection}
\makeatletter
\@addtoreset{figure}{problem}
\makeatother
\let\StandardTheFigure\thefigure
\let\vec\mathbf
\renewcommand{\thefigure}{\theproblem}
\def\putbox#1#2#3{\makebox[0in][l]{\makebox[#1][l]{}\raisebox{\baselineskip}[0in][0in]{\raisebox{#2}[0in][0in]{#3}}}}
     \def\rightbox#1{\makebox[0in][r]{#1}}
     \def\centbox#1{\makebox[0in]{#1}}
     \def\topbox#1{\raisebox{-\baselineskip}[0in][0in]{#1}}
     \def\midbox#1{\raisebox{-0.5\baselineskip}[0in][0in]{#1}}
\vspace{3cm}
\title{ASSIGNMENT 4}
\author{A.Tejasri}
\maketitle
\newpage
\bigskip
\renewcommand{\thefigure}{\theenumi}
\renewcommand{\thetable}{\theenumi}
%
Download all python codes from 
\begin{lstlisting}
https://github.com/tejasri3657/Assignment-4/blob/main/Assignment-4.py
\end{lstlisting}
%
and latex-tikz codes from 
%
\begin{lstlisting}
https://github.com/tejasri3657/Assignment-4/blob/main/main.tex
\end{lstlisting}
%
\section{Question No 2.39}
\ Find the equation of the plane through the intersection of the planes 
$
\myvec{3 & -1 & 2}\vec{x}=4
$
 and 
$
\myvec{1  &  1  &  1}\vec{x}=-2
$
and the point \myvec{2\\2\\1}.
%
\section{SOLUTION}
Given,
\begin{align}
\myvec{3 & -1 & 2}\vec{x}&=4 \label{2.0.1}
\\
\myvec{1 & 1 & 1}\vec{x}&=-2 \label{2.0.2}
\\
\vec{A}&=\myvec{2 \\ 2 \\ 1} \label{2.0.3}
\end{align}
%
Equation can be written as,
\begin{align}
\vec{n_1^T}\vec{x} &= c_1 \label{2.0.4}\\
\vec{n_2^T}\vec{x} &= c_2 \label{2.0.5}
\end{align}
Where,
\begin{align}
\vec{n_1}&=\myvec{3 \\ -1 \\ 2},\vec{n_2}=\myvec{1 \\ 1 \\ 1}
\\
c_1&=4,c_2=-2
\end{align}
Required equation of the plane containing \eqref{2.0.4} and \eqref{2.0.5} is,
\begin{align}
\vec{n_1^T}\vec{x} + \lambda\vec{n_2^T}\vec{x}  &= c_1+\lambda c_2  \\ 
\implies (\vec{n_1^T}+\lambda\vec{n_2^T})\vec{x} &= c_1+\lambda c_2 \label{2.0.9}
\end{align}
By substituting the intersection of the plane of the point \eqref{2.0.3}.so,
\begin{align}
\vec{A}(\vec{n_1^T}+\lambda\vec{n_2^T})\vec{x} &= c_1+\lambda c_2 \\
\lambda(\vec{A}\vec{n_2^T}-c_2)&=C_1-(\vec{A}\vec{n_1^T}) \\
\lambda&=\frac{(c_1-\vec{A}\vec{n_1^T})}{(\vec{A}\vec{n_2^T}-c_2)} \\
\implies\lambda &= \frac{-2}{7}
\end{align}
$\therefore $ By substituting $\lambda$,$n_1$,$n_2$,$c_1$,$c_2$ values in \eqref{2.0.9} we get required plane equation as,
\begin{align}
\myvec{19 & -9 & 12}\vec{x} = 32 
\end{align}
Plot of the plane :
\numberwithin{figure}{section}
\begin{figure}[H]
\centering
\includegraphics[width=\columnwidth]{Plane_plot.png}
\caption{Plot of the plane}
\label{Plot of the plane}
\end{figure}
\end{document}
